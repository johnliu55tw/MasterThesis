% Chap. 5 Conclusion
\chapter{結論與建議}
\section{結論}
第一代Yun-Trooper使用攝影機與影像處理進行無人導航,其運算量龐大且不具備避障功能。
而本論文開發的Yun-Trooper II使用GPS搭配光學雷達進行室外導航,利用GPS得知目前位置與目標位置的相對關係,
同時使用光學雷達在導航過程中迴避障礙物,達成完全的無人導航,且運算量較影像處理少。

在硬體方面,由於不需進行影像處理,本論文使用BeagleBone Black嵌入式電腦搭配GNU/Linux作業系統取代了原先的工業用主機板與Windows XP作業系統,
利用四驅遙控車的底盤開發了Yun-Trooper II自走車,取代了第一代Yun-Trooper。
此嵌入式電腦具有直接輸出PWM與數位I/O訊號的能力,因此簡化了週邊配備的需求,只需簡單的放大電路便可直接控制車輛動作。
因此,Yun-Trooper II不但降低了車輛的重量及體積,讓巡航速度增加,同時也降低了電力的耗損。

在路徑規劃演算法方面,本論文基於VFH+~\cite{Ulrich:1998:VFHPlus},開發了整合GPS導航的路徑規劃演算法,
同時解決了VFH+中的兩個問題:無候選角度及邊界誤判。

無候選角度發生在轉向方向較為極端的情況,原先的VFH+在此時會判定無方向可前進,造成導航停止。
因此本論文使用碰撞預測做為導航進行與否的指標,讓機器人能夠成功通過該環境。

本論文使用光學雷達做為環境感測器,因此簡化了VFH+的計算方式,但簡化的同時也造成了邊界誤判的問題。
因此本論文另外使用最接近之量測距離對轉向角度做出補償,解決此邊界誤判問題,同時不會大幅增加運算量。

在速度計算方面,由於Yun-Trooper II不具備速度感測器,但本身的運動速度對速度控制來說是相當重要的因素,
因此本論文利用光學雷達的資訊,估測出機器人與障礙物之間的速度,並將此項資訊加入速度控制,
讓機器人在快速接近障礙物時能夠有效的減速。

\section{建議}
本論文所使用的演算法為區域路徑規劃,並無全域路徑規劃,因此有時會將機器人導引至死路中,
而因為光學雷達的掃描範圍限制造成機器人無法向後方移動,讓機器人無法脫身,造成導航停止。
因此,可在後方再加裝一顆光學雷達,或是改變光學雷達的位置並加裝一旋轉平台,讓光學雷達在必要時能夠轉向,偵測後方環境。

在演算法方面,本論文所開發的路徑規劃演算法只針對當下的量測環境做規劃,並不會記錄路徑的選擇,因此無法追蹤機器人的導航過程,
若是導航至死路中也無法得知究竟是那一個決策造成如此的結果。因此,對路徑規劃的記錄也是另一個改進的重點,其資料結構也需要深入探討。

近年來,行動式機器人(Mobile Robot)的研究都朝統計機器人學(Probabilistic Robotics)的方向深入。
它使用統計學,建立了機器人的感知與控制系統的統計模型,並利用卡爾曼濾波器(Kalman Filter)等方式估測機器人的位置、速度等資訊,
建立一強建且能夠適應環境的路徑規劃演算法。而其中又以SLAM(Simultaneous Localization And Mapping,即時定位與地圖重建)
的概念最為特別,它希望能夠在未知的環境中,在定位自身位置的同時也能夠繪出環境的地圖。然而此項研究需要深入探討統計學,
同時對資料結構也必須相當有概念,才能夠將數學演算法轉化為實際的程式,應用到實際世界中,因此可能需要較長時間的研究。


